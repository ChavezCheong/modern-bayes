% Options for packages loaded elsewhere
\PassOptionsToPackage{unicode}{hyperref}
\PassOptionsToPackage{hyphens}{url}
%
\documentclass[
]{article}
\title{Data Wrangling in \texttt{R}}
\author{STA 360/602: Homework 1}
\date{Due Friday January 7 at 5 PM EDT}

\usepackage{amsmath,amssymb}
\usepackage{lmodern}
\usepackage{iftex}
\ifPDFTeX
  \usepackage[T1]{fontenc}
  \usepackage[utf8]{inputenc}
  \usepackage{textcomp} % provide euro and other symbols
\else % if luatex or xetex
  \usepackage{unicode-math}
  \defaultfontfeatures{Scale=MatchLowercase}
  \defaultfontfeatures[\rmfamily]{Ligatures=TeX,Scale=1}
\fi
% Use upquote if available, for straight quotes in verbatim environments
\IfFileExists{upquote.sty}{\usepackage{upquote}}{}
\IfFileExists{microtype.sty}{% use microtype if available
  \usepackage[]{microtype}
  \UseMicrotypeSet[protrusion]{basicmath} % disable protrusion for tt fonts
}{}
\makeatletter
\@ifundefined{KOMAClassName}{% if non-KOMA class
  \IfFileExists{parskip.sty}{%
    \usepackage{parskip}
  }{% else
    \setlength{\parindent}{0pt}
    \setlength{\parskip}{6pt plus 2pt minus 1pt}}
}{% if KOMA class
  \KOMAoptions{parskip=half}}
\makeatother
\usepackage{xcolor}
\IfFileExists{xurl.sty}{\usepackage{xurl}}{} % add URL line breaks if available
\IfFileExists{bookmark.sty}{\usepackage{bookmark}}{\usepackage{hyperref}}
\hypersetup{
  pdftitle={Data Wrangling in },
  pdfauthor={STA 360/602: Homework 1},
  hidelinks,
  pdfcreator={LaTeX via pandoc}}
\urlstyle{same} % disable monospaced font for URLs
\usepackage[margin=1in]{geometry}
\usepackage{graphicx}
\makeatletter
\def\maxwidth{\ifdim\Gin@nat@width>\linewidth\linewidth\else\Gin@nat@width\fi}
\def\maxheight{\ifdim\Gin@nat@height>\textheight\textheight\else\Gin@nat@height\fi}
\makeatother
% Scale images if necessary, so that they will not overflow the page
% margins by default, and it is still possible to overwrite the defaults
% using explicit options in \includegraphics[width, height, ...]{}
\setkeys{Gin}{width=\maxwidth,height=\maxheight,keepaspectratio}
% Set default figure placement to htbp
\makeatletter
\def\fps@figure{htbp}
\makeatother
\setlength{\emergencystretch}{3em} % prevent overfull lines
\providecommand{\tightlist}{%
  \setlength{\itemsep}{0pt}\setlength{\parskip}{0pt}}
\setcounter{secnumdepth}{-\maxdimen} % remove section numbering
\ifLuaTeX
  \usepackage{selnolig}  % disable illegal ligatures
\fi

\begin{document}
\maketitle

Today's agenda: Manipulating data objects; using the built-in functions,
doing numerical calculations, and basic plots; reinforcing core
probabilistic ideas.

\textbf{\emph{General instructions for homeworks}}: Please follow the
uploading file instructions according to the syllabus. You will give the
commands to answer each question in its own code block, which will also
produce plots that will be automatically embedded in the output file.
Each answer must be supported by written statements as well as any code
used. Your code must be completely reproducible and must compile.

\textbf{\emph{Advice}}: Start early on the homeworks and it is advised
that you not wait until the day of. While the professor and the TA's
check emails, they will be answered in the order they are received and
last minute help will not be given unless we happen to be free.

\textbf{\emph{Commenting code}} Code should be commented. See the Google
style guide for questions regarding commenting or how to write code
\url{https://google.github.io/styleguide/Rguide.xml}. No late homework's
will be accepted.

\textbf{\emph{R Markdown Test}}

\begin{enumerate}
\def\labelenumi{\arabic{enumi}.}
\setcounter{enumi}{-1}
\tightlist
\item
  Open a new R Markdown file; set the output to HTML mode and ``Knit''.
  This should produce a web page with the knitting procedure executing
  your code blocks. You can edit this new file to produce your homework
  submission.
\end{enumerate}

\textbf{\emph{Working with data}}

Total points on assignment: 10 (reproducibility) + 22 (Q1) + 9 (Q2) + 3
(Q3) = 44 points

Reproducibility component: 10 points.

\begin{enumerate}
\def\labelenumi{\arabic{enumi}.}
\tightlist
\item
  (22 points total, equally weighted) The data set \textbf{rnf6080.dat}
  records hourly rainfall at a certain location in Canada, every day
  from 1960 to 1980.
\end{enumerate}

\begin{enumerate}
\def\labelenumi{\alph{enumi}.}
\item
  Load the data set into R and make it a data frame called
  \texttt{rain.df}. What command did you use?
\item
  How many rows and columns does \texttt{rain.df} have? How do you know?
  (If there are not 5070 rows and 27 columns, you did something wrong in
  the first part of the problem.)
\item
  What command would you use to get the names of the columns of
  \texttt{rain.df}? What are those names?
\item
  What command would you use to get the value at row 2, column 4? What
  is the value?
\item
  What command would you use to display the whole second row? What is
  the content of that row?
\item
  What does the following command do?
\end{enumerate}

\begin{verbatim}
names(rain.df) <- c("year","month","day",seq(0,23))
\end{verbatim}

\begin{enumerate}
\def\labelenumi{\alph{enumi}.}
\setcounter{enumi}{6}
\tightlist
\item
  Create a new column called \texttt{daily}, which is the sum of the 24
  hourly columns.
\item
  Give the command you would use to create a histogram of the daily
  rainfall amounts. Please make sure to attach your figures in your .pdf
  report.
\item
  Explain why that histogram above cannot possibly be right.
\item
  Give the command you would use to fix the data frame.
\item
  Create a corrected histogram and again include it as part of your
  submitted report. Explain why it is more reasonable than the previous
  histogram.
\end{enumerate}

\textbf{\emph{Data types}}

\begin{enumerate}
\def\labelenumi{\arabic{enumi}.}
\setcounter{enumi}{1}
\tightlist
\item
  (9 points, equally weighted) Make sure your answers to different parts
  of this problem are compatible with each other.
\end{enumerate}

\begin{enumerate}
\def\labelenumi{\alph{enumi}.}
\tightlist
\item
  For each of the following commands, either explain why they should be
  errors, or explain the non-erroneous result.
\end{enumerate}

\begin{verbatim}
x <- c("5","12","7")
max(x)
sort(x)
sum(x)
\end{verbatim}

\begin{enumerate}
\def\labelenumi{\alph{enumi}.}
\setcounter{enumi}{1}
\tightlist
\item
  For the next two commands, either explain their results, or why they
  should produce errors.
\end{enumerate}

\begin{verbatim}
y <- c("5",7,12)
y[2] + y[3]
\end{verbatim}

\begin{enumerate}
\def\labelenumi{\alph{enumi}.}
\setcounter{enumi}{2}
\tightlist
\item
  For the next two commands, either explain their results, or why they
  should produce errors.
\end{enumerate}

\begin{verbatim}
z <- data.frame(z1="5",z2=7,z3=12)
z[1,2] + z[1,3]
\end{verbatim}

\begin{enumerate}
\def\labelenumi{\arabic{enumi}.}
\setcounter{enumi}{2}
\tightlist
\item
  (3 pts, equally weighted).
\end{enumerate}

a.) What is the point of reproducible code?

b.) Given an example of why making your code reproducible is important
for you to know in this class and moving forward.

c.) On a scale of 1 (easy) -- 10 (hard), how hard was this assignment.
If this assignment was hard (\(>5\)), please state in one sentence what
you struggled with.

\end{document}
